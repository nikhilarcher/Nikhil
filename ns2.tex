\documentclass[12pt]{book}
\usepackage{amsmath}
\begin{document}

\title{mathematical typesetting}
\author{ns}
\maketitle

\chapter{Equation}

\section{Inline Mode}
Linear equation: $ x+y=0 $ and $ x^{31} $

Meaning the next equation has no intiger solution:$x^n + y^n =z^n $

\section{displaced mode}
\underline{unnumbered}\\
In physics, the mass-energy equvalance is started by the equation $$ E=nc^2 $$ discopvered in 1905 by Albert Einstein. \\

\underline{numbered}\\

in natural units,

\begin{displaymath}
c=1
\end{displaymath}

The formula express tghe identity

\begin{equation}
E=m
\end{equation}

\section{Aligning Equation}

\begin{align*}
x + 3y + 4z &= 2 \\
3y - 4z &= 5 \\
3 &= 4
\end{align*}\\

\section{Fractions}

$\frac{x+3}{4}$

\section{subscript and superscript}

$ a_{11} $

$$ X + \frac{-b \pm \sqrt{b^2-4ac}}{2qa} $$

\section{special characters}

\{\& \}

\section{matrics}

1. plain\\

$$
\begin{bmatrix}
1 & 2 & 3\\
a & b & c
\end{bmatrix}
$$

\section{diwsplay style}

In-line math element $f(x)=\frac{1}{1+1}$ can be set with a different style: $
f(x)=\displaystyle\frac{1}{1+x}$

\end{document}